\documentclass{article}    
\usepackage[margin=0.5in]{geometry}
\usepackage{amsmath}
\usepackage{amsfonts}
\usepackage[us]{datetime}

\usepackage{pdfpages}
\usepackage{graphicx}
\usepackage{minted}
\usepackage{fontspec}
\newfontfamily\monaco{Iosevka Fixed}[NFSSFamily=IosevkaFamily]
\setminted{fontfamily=IosevkaFamily}
\usepackage{hyperref}
\hypersetup{colorlinks=true,linkcolor=blue,urlcolor=blue}

\setlength{\parindent}{0pt}

% SOME NOTES
%  - Start the auto-PDF-generator with `./lm`
%  - Use the T and F9 keys in Atril to minimize the toolbar and side panes
%

\newcommand{\classname}{EN.525.627: Digital Signal Processing}
\newcommand{\classprof}{Prof. Brian Jennison}

\title{\vspace{64pt} \bf{DSP Final Project} \\ \textnormal{BPSK in audio steganography: a noise-resistant encoding scheme}}
\newdate{date}{4}{12}{2024}

% Hack this field to get some extra text in the title
% https://tex.stackexchange.com/questions/63384/add-additional-text-on-title-page
\author{\bf{Jonathan Piland}}
\date{\parbox{\linewidth}{\centering
  \endgraf\bigskip
  \classname\endgraf
  \classprof\endgraf\medskip
  Whiting School of Engineering \endgraf
Johns Hopkins University\endgraf\bigskip\displaydate{date}}}

% OPTIONAL use of alphabetic characters for numbering... comment if unwanted
% (usually used for EN.525.627)
% \renewcommand\thesection{\arabic{section}.}
% \renewcommand\thesubsection{\arabic{section}.\alph{subsection}.}
%

\begin{document}
\maketitle

NOTE: All code, audio samples, and text files are available at 
\url{https://github.com/jpiland16/dsp-final-project}.

\newpage
\section{Introduction}
% Describe the project that you performed. What problem were 
% you trying to solve? Why is it important?
Hiding a message in an ordinary-seeming object is a task that happens quite
frequently in the digital age. For example, the producer of a digital video
may desire to hide a message within the video file to indicate that they are the rightful
owner of the content in the video. Alternatively, an individual in an intelligence
role might want to transfer a message to another individual via a covert medium
that appears to be a normal music file, for example.

\vspace{6pt}

This project explores the possibility of hiding data within an audio file, a process
that is known as \textit{audio steganography}. There are some methods of covertly inserting 
a message into an audio signal that might be
more susceptible to detection or destruction than others; this project explores
two of those possibilities and proves the superiority of one.

\section{Approach}
% What was the general approach that you took? What other
% approaches might be considered to address the same problem? What are
% advantages and disadvantages to your approach?

In this section, the general approaches to the problem are considered. First,
LSB encoding is discussed, and then the alternative of binary phase-shift keying
(which was the focus of this project) is introduced.

\subsection{LSB encoding}

One of the simplest ways to covertly insert secret information into an audio
signal is to add this data to the least significant bit (LSB) of the audio data.
For a 32-bit waveform, this has the effect of altering each sample of the data
by one part in $2^31$ or less, i.e. one part in about 2 billion. This change is
completely imperceptible to the human auditory system, and is essentially
invisible to spectrum analysis with the naked eye. For example, the below shows
the (non-normalized) spectrum of 8 seconds of audio before the secret message is
added to the LSBs. 

\begin{center}
  \includegraphics[width=3in]{img/01_original_spectrum.png} \\
  \textbf{Figure 1: Spectrum of signal used for hiding secret message, before inserting data}
\end{center}

After adding the data to the LSB of each sample, the difference in the spectra
between the secret audio and the unmodified audio is eight orders of magnitude
smaller than the maximum amplitude of the original spectrum:

\begin{center}
  \includegraphics[width=3in]{img/02_difference_spectrum.png} \\
  \textbf{Figure 2: Difference in spectra between original and secret audio}
\end{center}

Using this technique, data can be hidden within
the audio file at a rate of 1 bit per sample per second, so at a sample rate
of 44.1 kHz, data can be hidden at a rate of about 5.38 kB/s. However, this
encoding is extremely susceptible to noise - a result which will be shown in the
next section (Results).

\subsection{Binary phase-shift keying}
Binary phase-shift keying (BPSK) is a subset of phase-shift keying that uses
just two phases of a carrier wave to encode and transmit data. To better
understand the transmission system developed in this project, the following
diagram from Schaumont is helpful:

\begin{center}
  \includegraphics[width=7in]{img/commsystem.png} \\
  \textbf{Figure 3: Simplified view of a data communication system} \\
  Figure reproduced from \href{https://schaumont.dyn.wpi.edu/ece4703b21/lecture9.html}{Schaumont}.
\end{center}

An adaptation of the BPSK modulator and demodulator was developed based on this
diagram and the material present in Schaumont. The function of this system can
be explained briefly as follows: first, a series of bits must be transformed
into a set of symbols that can be used to modulate the wave. In this case, since
we are using BPSK, the symbols chosen can simply be $-1$ and $+1$, since
multiplying a carrier wave by these symbols will generate a 180-degree phase
separation between symbols. Second, a pulse shaping filter must be applied to
the symbols (which are upsampled to the sampling rate of the channel) in order
to reduce the required bandwidth of the signal and allow for easier detection.
Finally, the shaped signal (which is often called a `baseband' waveform) is
multiplied by a carrier wave to create the modulated signal.

\vspace{6pt}

To receive the transmitted bits, this operation is essentially performed in
reverse: the carrier wave must be detected and removed, and a matched filter can
be used to recover the symbols. Finally, the symbols must be decoded back into
bits to recover the original data.

\vspace{6pt}

The pulse-shaping filter chosen for this implementation was a root-raised cosine
filter. The carrier frequency selected was seven times the symbol frequency,
which means the carrier frequency is 19.29375 kHz with a choice of 16 samples per
symbol and an audio sampling rate of 44.1 kHz. ($19.29375 = 7/16 \times 44.1$)

\subsection{Selected samples}

The audio selected for use in this project is ``Blinding Lights'' by The Weeknd,
and the data selected for embedding is the text of 
\textit{The Lion, the Witch, and the Wardrobe} by C. S. Lewis. There is no
particular reason for this, other than the fact that these are things that the
author would not mind hearing or reading repeatedly when working on the project.

\section{Results}
% This section will likely be the largest section in your report. It should
% include figures, tables, graphs, etc. summarizing the results of your project.

\subsection{Implementation of LSB encoding}

LSB encoding was accomplished by converting the ASCII message into a bitstream
using the least-significant-bit-first method. The eight bits representing each
character were inserted into the LSBs of eight consecutive audio samples as
shown in the code below, where $A$ is the audio to be modified:

\inputminted[xleftmargin=24pt, linenos=true, breaklines, firstline=26, lastline=33]{matlab}{src/encode_message_LSB.m}

However, as discussed previously, this encoding is extremely susceptible to
the addition of noise. Adding a noise at a level of $-190$ dB (nearly 10 orders
of magnitude smaller than the largest signal) is enough to render this encoding
essentially useless. The added noise is imperceptible to the human ear (try
listening to audio files 01 and 02 for comparison) and could be a simple
technique to destroy any data hidden with LSB encoding. The below MATLAB code
shows how the noise impacts the signal:

\inputminted[xleftmargin=24pt, linenos=true, breaklines, firstline=38, lastline=39]{matlab}{project_code.m}

With this change, almost 90\% of the characters received were incorrect. For
comparison, the expected text to be returned looked like this:

\begin{minted}[xleftmargin=24pt]{text}
ONCE there were four children whose names were Peter, Susan, Edmund 
and Lucy. This story is about ...
\end{minted}

but with the added $-190$ dB noise, we get this:

\begin{minted}[xleftmargin=24pt]{text}
kFBG–thÕÿ5!t$Sð,fï1ie\(m©mH1NÄ|øj?1yoqmeq(t-se1ômdEJ,?Ï\}Sy.M)TF'÷oÀx²b,&#H-ªv. DxI³2sôoðù4Yv áiE\}dOU ...
\end{minted}

A possible solution to this might be to repeat each bit a certain number of
times. To compare this result to the use of BPSK, where 16 samples per
symbol was selected, each bit was repeated 16 times. With this repetition, the
effect on the spectrum was still minimal, as shown below. However, an appearance
of a sinc envelope (with period Fs/16) is now clearly visible, since the
repetition of bits creates small rectangles in the signal with width 16/Fs. This
sinc function is plotted on the graph below.

\begin{center}
  \includegraphics[width=3in]{img/04_difference_spectrum_repeat_sinc.png} \\
  \textbf{Figure 3: Difference in spectra between original and secret audio when repeating bits}
\end{center}

Using the repeats at this level of noise with a decoder that takes the majority
consensus of the bits generates a significant improvement at this level of
noise, but if the noise is increased even slightly, the data is once again
destroyed. Below is the output of the decoder when using 16 repeats at the same
noise level previously applied:

\begin{minted}[xleftmargin=24pt]{text}
ONCE tHere were four shaldren whose nam%s wer% Peper, Susan, Edmund 
and Lucy. THas stor9 is about s
\end{minted}

With this change, 89\% of the bits were decoded correctly with the noise
applied. However, this noise was incredibly small, as shown below, and this 
necessitates an improved method of covertly encoding the data into the audio
signal.

\begin{center}
  \includegraphics[width=3in]{img/05_difference_spectrum_repeat_sinc_noisy.png} \\
  \textbf{Figure 4: Difference in spectra between original and secret audio with repeats and noise}
\end{center}

\subsection{Implementation of BPSK modulator/demodulator}

As previously discussed, the BPSK modulator and demodulator were implemented
using a three-stage model of the transmitter and receiver: encoding, pulse
shaping, and modulation at the transmitter followed by demodulation, pulse
matching, and decoding at the receiver.

\vspace{6pt}

Encoding the binary data as symbols was relatively straightforward, and was
conducted in a similar manner to that when converting the message to bits for
LSB encoding. In this case, however, the final output was the symbols $-1$ and
$+1$ rather than the binary values of 0 and 1. This encoding is demonstrated
by the below MATLAB code.

\inputminted[xleftmargin=24pt, linenos=true, breaklines, firstline=32, lastline=39]{matlab}{src/encode_message_BPSK.m}

In addition to this encoding, a synchronization sequence which consisted of 16
`'$+1$' symbols followed by an alternating sequence of length 16 was added to
each group of characters. This aids in recognition of the signal at the
demodulator. Character groups of length 32 were used, and therefore there is an
overhead of 16 symbols per 256 symbols of characters, i.e. 12.5\% overhead.

\inputminted[xleftmargin=24pt, linenos=true, breaklines, firstline=41, lastline=49]{matlab}{src/encode_message_BPSK.m}

The next step, which is arguably the most interesting from a DSP perspective,
is the upsampling and pulse shaping of the signal into a baseband waveform.
Immediately after upsampling, the waveform contains high-frequency replicated
components due to the addition of ($N - 1$) zeros between each symbol. This
unnecessarily increases the signal bandwidth, and should be reduced.

\vspace{6pt}

Before the application of the pulse-shaping filter, the upsampled signal
has an extremely wide bandwidth:

\begin{center}
  \includegraphics[width=3in]{img/06_symbols.png} \hspace{6pt}
  \includegraphics[width=3in]{img/07_fft_symbols_up.png} \\
  \textbf{Figure 5: Start of symbols to transmit (left) and Fourier transform of upsampled symbols (right)}
\end{center}

To reduce the bandwidth of the signal, a root-raised cosine filter was chosen
and implemented in MATLAB using \texttt{rcosdesign}. The parameters selected
were a rolloff factor of 0.25 (additional bandwidth beyond the signal
bandwidth) and a filter cutoff spanning 11 symbol periods. The relevant
tradeoffs here are as follows:

\begin{enumerate}
  \item A larger rolloff means that the height of the pulse approaches zero more quickly, which means the pulse is closer to an all-pass filter (delta function). This requires a smaller filter to implement, but comes at the expense of higher bandwidth.
  \item A smaller rolloff means that the resulting signal bandwidth will be less, but a longer filter is required.
  \item Clipping the filter to a certain number of symbol periods means there may be some ``jumps" in the signal or imperfect interpolation.
\end{enumerate}

After creating the filter using \texttt{rcosdesign}, the resulting pulse had
the following shape and Fourier transform:

\begin{center}
  \includegraphics[width=3in]{img/08_rcpulse.png} \hspace{6pt}
  \includegraphics[width=3in]{img/09_rcpulse_fft.png} \\
  \textbf{Figure 6: Pulse shape and frequency response of root-raised cosine filter}
\end{center}

This FIR filter has a length of 177 and a constant group delay of 88 samples.
Its cutoff frequency is, as expected, slightly more than 1/16th of the Nyquist
frequency, which means its cutoff frequency is about 1400 Hz.

\vspace{6pt}

After applying the RRC pulse to the upsampled baseband signal, the bandwidth of
the signal is greatly reduced, as shown in the figure below. (The colored
rectangles in the graph at right identify the symbol being encoded - green is
$+1$ while red is $-1$.)

\begin{center}
  \includegraphics[width=3in]{img/12_baseband.png} \hspace{6pt}
  \includegraphics[width=3in]{img/11_fft_symbols_up.png} \\
  \textbf{Figure 7: Start of symbols to transmit (left) and Fourier transform of pulse-shaped symbols (right)}
\end{center}

An interesting plot which I learned about during this project is the eye diagram,
which superimposes all symbol waveforms over top of each other. I have colorized
the plot so it is clear which waveforms correspond to the $+1$ (green) and $-1$
(red) symbols. The reason that symbols can have different waveforms is due to
the interpolation signal from adjacent symbols that appears due to the raised
cosine filter.

\begin{center}
  \includegraphics[width=3in]{img/12_eyediagram.png} \\
  \textbf{Figure 8: Eye diagram for baseband signal}
\end{center}

After modulation, the frequency spectrum is shifted to the high frequency ends of
the spectrum. A high-frequency carrier was chosen so this modulation would be
inaudible to the human ear.

\begin{center}
  \includegraphics[width=3in]{img/14_mc.png} \hspace{6pt}
  \includegraphics[width=3in]{img/13_mc_fft.png} \\
  \textbf{Figure 9: Start of modulated carrier (left) and Fourier transform of waveform (right)}
\end{center}

The final step in this system is to add the background audio as a cover for our
ultrasonic signal. This does not significantly affect the high-frequency content
of the signal, as shown below.


\begin{center}
  \includegraphics[width=3in]{img/15_bpsk_fft.png} \\
  \textbf{Figure 10: FFT of signal with cover audio}
\end{center}

To demodulate the signal, a Costas loop was implemented based on discussions
found in Schaumont. The Costas loop detects the baseband signal by using the
error between the in-phase (I) and out-of-phase (Q) branches to correct the
phase of the local oscillator. The matched RRC pulse is applied in determining this
phase error. A method of trial and error was employed in order
to find the correct tuning factor for the phase correction.

\vspace{6pt}

After the Costas loop has locked on to the signal, the demodulated baseband signal
can be found at the output of the I branch of the loop. 

\vspace{6pt}

One major improvement that was made to this loop was the use of an elementwise
multiplication and sum to determine the instantaneous error as opposed to selecting
a single element from the convolution of the RRC pulse with the entire loop output
up to that point (which is the method that was proposed in Schaumont.)

\vspace{6pt}

After recovering the baseband signal, the symbols can be decoded if the symbol
offset is known relative to the timing of the Costas loop. While we could
accurately predict the symbol placement based on the group delays of the matched
filter, if the system was receiving a signal asynchronously from a different
device, this timing might not be known. To address this potential issue, a simple
solution to recover symbol timing was created. The average deviation over a sample
period was taken for each of the 16 possible timing offsets, and the maximum
absolute value of this average deviation indicates the timing which is most
closely aligned with symbol edges. A plot of the scores for different offsets
is shown below for reference.


\begin{center}
  \includegraphics[width=3in]{img/16_scores.png} \\
  \textbf{Figure 11: Use of deviation score to recover symbol timing}
\end{center}

After timing recovery, an eye diagram can be constructed at the receiver:

\begin{center}
  \includegraphics[width=3in]{img/17_eyediagram1.png} \\
  \textbf{Figure 12: Eye diagram for baseband signal after recovery}
\end{center}

This diagram shows a clear distinction between the symbols, and indicates that the 
center region of a symbol is an ideal region for deciding which of the two symbols
is present in a signal.

\vspace{6pt}

After recovering the symbol timing and extracting symbols from the received
baseband signal, the remaining task was to identify the synchronization sequence
and convert the received symbols into ASCII characters. The code that performs
this decoding is shown below. Importantly, the symbols have to be checked for
a 180-degree phase flip, since the Costas loop has an ambiguity in the
detected phase. The known signs of the synchronization signal can be used to
identify a phase flip.

\inputminted[xleftmargin=24pt, linenos=true, breaklines, firstline=107, lastline=132]{matlab}{src/decode_message_BPSK.m}

\subsection{Major results}
Using the system described above, the secret message can be decoded
with \textbf{complete accuracy, even in the presence of noise}. The system
was tested and confirmed to be resilient to noise levels up to one-tenth
of the overall signal level (-20 dB). This is an improvement on the order of
nine orders of magnitude in the amount of noise tolerated when compared to the
LSB encoding method, and the sound is still completely inaudible to the human ear.
When listening to the BPSK-modulated signal with noise (see audio sample 06), the noise is clearly
clearly audible to the human ear, but the signal is still undetectable. This means that the secret data could be stored in an audio file undetected and
used to secretly convey a message, even if noise was added to the signal to
attempt to destroy the message.
 
\vspace{6pt}

The eye diagram below shows how the eye diagram changes through the addition of
noise, but there is still enough ``whitespace'' to correctly identify each symbol.

\begin{center}
  \includegraphics[width=3in]{img/18_eyediagram2.png} \\
  \textbf{Figure 13: Eye diagram for baseband signal after recovery in noise}
\end{center}


\section{Summary and Future Work}
% What did you learn while completing this
% project? If you had additional time, what else would you have done?

\subsection{Knowledge gained}

One topic I learned about this project was the use of root-raised cosine filters
to perform pulse shaping and pulse matching. This is something I did not have
previous experience with. I also learned about the application of a Costas loop to recover the phase
changes in a modulated carrier, and I learned that there is an inherent ambiguity
in phase that may have to be corrected after identifying symbols.

\subsection{Future work}

One thing I had hoped to get working in this project that proved more difficult
than I had anticipated was to transmit a secret message from one device to 
another using this method. I had hoped that the ultrasonic method of transmission
would work well, but it seemed to be hard to detect using the Costas loop.
Perhaps another keying method would be applicable here.

\vspace{6pt}

After many tests of trying to make this transmission ``over the air'' i.e. using
sound waves, I was finally successful at decoding a message with 630 samples per
symbol and a carrier frequency of 280 Hz that was played on my phone and recorded
on my computer. The ``secret'' signal was incredibly loud and easy to detect,
however, so I did not consider that a success. The recording which I was able
to detect is included here; it is called \texttt{2024-11-30\_16-58-45.wav}.

\vspace{6pt}

I also think it would be interesting to apply an error-correcting code to this
system and measure the increased reliability.

\subsection{Summary}
This project was an excellent opportunity for me to practice my skills applying
incredibly useful MATLAB commands, such as filter design, frequency domain analysis,
and simulation of communications systems. The success of using BPSK to hide a message
in an audio signal has piqued my interest in learning about other keying methods
and communication systems.

\section{References}

\begin{enumerate}
  \item Shilpi Mishra, Virendra Kumar Yadav, et al, ``Audio Steganography Techniques: A
Survey'' Advances in Computer and Computational Sciences, Advances in Intelligent
Systems and Computing 554, S.K. Bhatia et al. (eds.).
  \item Schaumont, Patrick. ``Digital Modulation.'' \textit{Worcester Polytechnic University}, 2021. \url{https://schaumont.dyn.wpi.edu/ece4703b21/lecture9.html}
\end{enumerate}

\newpage
\section{Appendix: MATLAB Code}
\subsection{\texttt{project\_code.m}}
\inputminted[xleftmargin=24pt, linenos=true, breaklines]{matlab}{project_code.m}

\newpage
\subsection{\texttt{constants.m}}
\inputminted[xleftmargin=24pt, linenos=true, breaklines]{matlab}{src/constants.m}

\newpage
\subsection{\texttt{encode\_message\_LSB.m}}
\inputminted[xleftmargin=24pt, linenos=true, breaklines]{matlab}{src/encode_message_LSB.m}

\newpage
\subsection{\texttt{decode\_message\_LSB.m}}
\inputminted[xleftmargin=24pt, linenos=true, breaklines]{matlab}{src/decode_message_LSB.m}

\newpage
\subsection{\texttt{encode\_message\_LSB\_repeat.m}}
\inputminted[xleftmargin=24pt, linenos=true, breaklines]{matlab}{src/encode_message_LSB_repeat.m}

\newpage
\subsection{\texttt{decode\_message\_LSB\_repeat.m}}
\inputminted[xleftmargin=24pt, linenos=true, breaklines]{matlab}{src/decode_message_LSB_repeat.m}

\newpage
\subsection{\texttt{encode\_message\_BPSK.m}}
\inputminted[xleftmargin=24pt, linenos=true, breaklines]{matlab}{src/encode_message_BPSK.m}

\newpage
\subsection{\texttt{decode\_message\_BPSK.m}}
\inputminted[xleftmargin=24pt, linenos=true, breaklines]{matlab}{src/decode_message_BPSK.m}

\newpage
\subsection{\texttt{plot\_fft\_with\_Fs.m}}
\inputminted[xleftmargin=24pt, linenos=true, breaklines]{matlab}{src/plot_fft_with_Fs.m}

\newpage
\subsection{\texttt{plot\_graphs.m}}
\inputminted[xleftmargin=24pt, linenos=true, breaklines]{matlab}{plot_graphs.m}

\end{document}
