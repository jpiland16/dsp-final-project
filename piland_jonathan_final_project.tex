\documentclass{article}    
\usepackage[margin=0.5in]{geometry}
\usepackage{amsmath}
\usepackage{amsfonts}
\usepackage[us]{datetime}

\usepackage{pdfpages}
\usepackage{graphicx}
\usepackage{minted}
\usepackage{fontspec}
\newfontfamily\monaco{Iosevka Fixed}[NFSSFamily=IosevkaFamily]
\setminted{fontfamily=IosevkaFamily}
\usepackage{hyperref}
\hypersetup{colorlinks=true,linkcolor=blue,urlcolor=blue}

\setlength{\parindent}{0pt}

% SOME NOTES
%  - Start the auto-PDF-generator with `./lm`
%  - Use the T and F9 keys in Atril to minimize the toolbar and side panes
%

\newcommand{\classname}{EN.525.627: Digital Signal Processing}
\newcommand{\classprof}{Prof. Brian Jennison}

\title{\vspace{64pt} \bf{DSP Final Project} \\ \textnormal{BPSK in audio steganography: a noise-resistant encoding scheme}}
\newdate{date}{4}{12}{2024}

% Hack this field to get some extra text in the title
% https://tex.stackexchange.com/questions/63384/add-additional-text-on-title-page
\author{\bf{Jonathan Piland}}
\date{\parbox{\linewidth}{\centering
  \endgraf\bigskip
  \classname\endgraf
  \classprof\endgraf\medskip
  Whiting School of Engineering \endgraf
Johns Hopkins University\endgraf\bigskip\displaydate{date}}}

% OPTIONAL use of alphabetic characters for numbering... comment if unwanted
% (usually used for EN.525.627)
% \renewcommand\thesection{\arabic{section}.}
% \renewcommand\thesubsection{\arabic{section}.\alph{subsection}.}
%

\begin{document}
\maketitle

NOTE: All code, audio samples, and text files are available at 
\url{https://github.com/jpiland16/dsp-final-project}.

\newpage
\section{Introduction}
% Describe the project that you performed. What problem were 
% you trying to solve? Why is it important?

\section{Approach}
% What was the general approach that you took? What other
% approaches might be considered to address the same problem? What are
% advantages and disadvantages to your approach?

In this section, the general approaches to the problem are considered. First,
LSB encoding is discussed, and then the alternative of binary phase-shift keying
(which was the focus of this project) is introduced.

\subsection{LSB encoding}

One of the simplest ways to covertly insert secret information into an audio
signal is to add this data to the least significant bit (LSB) of the audio data.
For a 32-bit waveform, this has the effect of altering each sample of the data
by one part in $2^31$ or less, i.e. one part in about 2 billion. This change is
completely imperceptible to the human auditory system, and is essentially
invisible to spectrum analysis with the naked eye. For example, the below shows
the (non-normalized) spectrum of 8 seconds of audio before the secret message is
added to the LSBs. 

\begin{center}
  \includegraphics[width=3in]{img/01_original_spectrum.png} \\
  \textbf{Figure 1: Spectrum of signal used for hiding secret message, before inserting data.}
\end{center}

After adding the data to the LSB of each sample, the difference in the spectra
between the secret audio and the unmodified audio is eight orders of magnitude
smaller than the maximum amplitude of the original spectrum:

\begin{center}
  \includegraphics[width=3in]{img/02_difference_spectrum.png} \\
  \textbf{Figure 2: Difference in spectra between original and secret audio.}
\end{center}

Using this technique, data can be hidden within
the audio file at a rate of 1 bit per sample per second, so at a sample rate
of 44.1 kHz, data can be hidden at a rate of about 5.38 kB/s. However, this
encoding is extremely susceptible to noise - a result which will be shown in the
next section (Results).

\newpage
\subsection{Binary phase-shift keying}
Binary phase-shift keying (BPSK) is a subset of phase-shift keying that uses
just two phases of a carrier wave to encode and transmit data. To better
understand the transmission system developed in this project, the following
diagram from Schaumont is helpful:

\begin{center}
  \includegraphics[width=7in]{img/commsystem.png} \\
  \textbf{Figure 3: Simplified view of a data communication system.} \\
  Figure reproduced from \href{https://schaumont.dyn.wpi.edu/ece4703b21/lecture9.html}{Schaumont}.
\end{center}

An adaptation of the BPSK modulator and demodulator was developed based on this
diagram and the material present in Schaumont. The function of this system can
be explained briefly as follows: first, a series of bits must be transformed
into a set of symbols that can be used to modulate the wave. In this case, since
we are using BPSK, the symbols chosen can simply be $-1$ and $+1$, since
multiplying a carrier wave by these symbols will generate a 180-degree phase
separation between symbols. Second, a pulse shaping filter must be applied to
the symbols (which are upsampled to the sampling rate of the channel) in order
to reduce the required bandwidth of the signal and allow for easier detection.
Finally, the shaped signal (which is often called a `baseband' waveform) is
multiplied by a carrier wave to create the modulated signal.

\vspace{6pt}

To receive the transmitted bits, this operation is essentially performed in
reverse: the carrier wave must be detected and removed, and a matched filter can
be used to recover the symbols. Finally, the symbols must be decoded back into
bits to recover the original data.

\vspace{6pt}

The pulse-shaping filter chosen for this implementation was a root-raised cosine
filter. The carrier frequency selected was seven times the symbol frequency,
which means the carrier frequency is 19.29375 kHz with a choice of 16 samples per
symbol and an audio sampling rate of 44.1 kHz. ($19.29375 = 7/16 \times 44.1$)

\subsection{Selected samples}

The audio selected for use in this project is ``Blinding Lights'' by The Weeknd,
and the data selected for embedding is the text of 
\textit{The Lion, the Witch, and the Wardrobe} by C. S. Lewis. There is no
particular reason for this, other than the fact that these are things that the
author would not mind hearing or reading repeatedly when working on the project.

\section{Results}
% This section will likely be the largest section in your report. It should
% include figures, tables, graphs, etc. summarizing the results of your project.

\subsection{Implementation of LSB encoding}

LSB encoding was accomplished by converting the ASCII message into a bitstream
using the least-significant-bit-first method. The eight bits representing each
character were inserted into the LSBs of eight consecutive audio samples as
shown in the code below, where $A$ is the audio to be modified:

\inputminted[xleftmargin=24pt, linenos=true, breaklines, firstline=26, lastline=33]{matlab}{src/encode_message_LSB.m}

However, as discussed previously, this encoding is extremely susceptible to
the addition of noise. Adding a noise at a level of $-190$ dB (nearly 10 orders
of magnitude smaller than the largest signal) is enough to render this encoding
essentially useless. The added noise is imperceptible to the human ear (try
listening to audio files 01 and 02 for comparison) and could be a simple
technique to destroy any data hidden with LSB encoding. The below MATLAB code
shows how the noise impacts the signal:

\inputminted[xleftmargin=24pt, linenos=true, breaklines, firstline=38, lastline=39]{matlab}{project_code.m}

With this change, almost 90\% of the characters received were incorrect. For
comparison, the expected text to be returned looked like this:

\begin{minted}[xleftmargin=24pt]{text}
ONCE there were four children whose names were Peter, Susan, Edmund 
and Lucy. This story is about ...
\end{minted}

but with the added $-190$ dB noise, we get this:

\begin{minted}[xleftmargin=24pt]{text}
kFBG–thÕÿ5!t$Sð,fï1ie\(m©mH1NÄ|øj?1yoqmeq(t-se1ômdEJ,?Ï\}Sy.M)TF'÷oÀx²b,&#H-ªv. DxI³2sôoðù4Yv áiE\}dOU ...
\end{minted}

A possible solution to this might be to repeat each bit a certain number of
times. To compare this result to the use of BPSK, where 16 samples per
symbol was selected, each bit was repeated 16 times. With this repetition, the
effect on the spectrum was still minimal, as shown below. However, an appearance
of a sinc envelope (with period Fs/16) is now clearly visible, since the
repetition of bits creates small rectangles in the signal with width 16/Fs. This
sinc function is plotted on the graph below.

\begin{center}
  \includegraphics[width=3in]{img/04_difference_spectrum_repeat_sinc.png} \\
  \textbf{Figure 3: Difference in spectra between original and secret audio when repeating bits.}
\end{center}

Using the repeats at this level of noise with a decoder that takes the majority
consensus of the bits generates a significant improvement at this level of
noise, but if the noise is increased even slightly, the data is once again
destroyed. Below is the output of the decoder when using 16 repeats at the same
noise level previously applied:

\begin{minted}[xleftmargin=24pt]{text}
ONCE tHere were four shaldren whose nam%s wer% Peper, Susan, Edmund 
and Lucy. THas stor9 is about s
\end{minted}

With this change, 89\% of the bits were decoded correctly with the noise
applied. However, this noise was incredibly small, as shown below, and this 
necessitates an improved method of covertly encoding the data into the audio
signal.

\begin{center}
  \includegraphics[width=3in]{img/05_difference_spectrum_repeat_sinc_noisy.png} \\
  \textbf{Figure 4: Difference in spectra between original and secret audio with repeats and noise.}
\end{center}

\subsection{Implementation of BPSK modulator/demodulator}

\section{Summary and Future Work}
% What did you learn while completing this
% project? If you had additional time, what else would you have done?

\section{References}

\newpage
\section{Appendix: MATLAB Code}
\subsection{\texttt{project\_code.m}}
\inputminted[xleftmargin=24pt, linenos=true, breaklines]{matlab}{project_code.m}

\subsection{\texttt{encode\_message\_LSB.m}}
\inputminted[xleftmargin=24pt, linenos=true, breaklines]{matlab}{src/encode_message_LSB.m}

\subsection{\texttt{decode\_message\_LSB.m}}
\inputminted[xleftmargin=24pt, linenos=true, breaklines]{matlab}{src/decode_message_LSB.m}

\subsection{\texttt{encode\_message\_LSB\_repeat.m}}
\inputminted[xleftmargin=24pt, linenos=true, breaklines]{matlab}{src/encode_message_LSB_repeat.m}

\subsection{\texttt{decode\_message\_LSB\_repeat.m}}
\inputminted[xleftmargin=24pt, linenos=true, breaklines]{matlab}{src/decode_message_LSB_repeat.m}

\subsection{\texttt{encode\_message\_BPSK.m}}
\inputminted[xleftmargin=24pt, linenos=true, breaklines]{matlab}{src/encode_message_BPSK.m}

\subsection{\texttt{decode\_message\_BPSK.m}}
\inputminted[xleftmargin=24pt, linenos=true, breaklines]{matlab}{src/decode_message_BPSK.m}

\subsection{\texttt{plot\_fft\_with\_Fs.m}}
\inputminted[xleftmargin=24pt, linenos=true, breaklines]{matlab}{src/plot_fft_with_Fs.m}

\subsection{\texttt{plot\_graphs.m}}
\inputminted[xleftmargin=24pt, linenos=true, breaklines]{matlab}{plot_graphs.m}

\end{document}
